

% Glossary entries
\newglossaryentry{def:stuk}{name=Stuk,text=stuk,plural=stukken, description={Een
archiefstuk, boek, brochure, tijdschrift, of andere ge\"indexeerde eenheid uit
het archief of bibliotheek van het IISG. Indien een archief niet ge\"indexeerd
is, wordt het hele archief als \'e\'en stuk beschouwd. Alle stukken zijn
analoog, d.w.z. fysiek aanwezig, tenzij er wordt gesproken over digitale
stukken. Zowel de algemene metadata over een stuk, als een specifiek exemplaar
kan een `stuk' worden genoemd. Voorbeeld van onderscheid: Er zijn een aantal
exemplaren van een boek aanwezig. De contactinformatie, titel, inzage
restricties e.d. staan opgeslagen in de algemene metadata van het stuk. De
locatie informatie, gebruiksrestrictie (open/dicht) en status (uitgeleend of
niet) zijn per exemplaar opgeslagen}}
\newglossaryentry{def:hrb}{name=Hoog Resolutie Bestand,text=hoog resolutie
bestand,plural=hoge resolutie bestanden,description={Een afbeelding met hoge
resolutie/kwaliteit. Exacte resolutie/kwaliteit is door een extern systeem
bepaald (De Shared Object Repository)}}
\newglossaryentry{def:pdf}{name=PDF,description={Portable Document Format; een
veelgebruikt bestandsformaat om documenten in op te slaan}}
\newglossaryentry{def:ftp}{name=FTP,description={File Transfer Protocol; een
gestandaardiseerd protocol om bestanden uit te wisselen tussen verschillende
computers}}
\newglossaryentry{def:use-case}{name=Use Case,text=use case,description={Een
bepaalde sequentie van handelingen die de interactie tussen een gebruiker en een
systeem weergeeft}}
\newglossaryentry{def:ui}{name=User Interface,text=user
interface,description={Deel van het systeem dat de interactie tussen de
gebruiker en het systeem mogelijk maakt}}
\newglossaryentry{def:fr}{name=Functionele Requirement,text=functionele
requirement,description={Eis die beschrijft wat het systeem moet kunnen. Dit
type eis is direct af te leiden uit de probleemstelling}}
\newglossaryentry{def:nfr}{name=Niet-Functionele Requirement,text=niet-functionele
requirement,description={Eis die de randvoorwaarden aan het
systeem die niet direct uit de probleemstelling zijn af te leiden beschrijft}}
\newglossaryentry{def:bezoeker}{name=Bezoeker,text=bezoeker,description={Een persoon
die te gast is bij het IISG om danwel stukken in te zien, danwel reproducties te
bestellen. De bezoeker kan zowel via het internet, als op locatie een service
van het IISG gebruiken}}
\newglossaryentry{def:medewerker}{name=Medewerker,text=medewerker,description={Een
magazijn-,
studiezaal-, of reproductiemedewerker}}
\newglossaryentry{def:studiezaalmedewerker}{name=Studiezaalmedewerker,text=studiezaalmedewerker,
description={Een medewerker van de studiezaal die bezoekers te woord staat en
stukken uitleent ter inzage}}
\newglossaryentry{def:magazijnmedewerker}{name=Magazijnmedewerker,text=magazijnmedewerker,
description={Een medewerker die stukken van en naar het magazijn brengt}}
\newglossaryentry{def:reproductiemedewerker}{name=Reproductiemedewerker,text=reproductiemedewerker,
description={Een medewerker die zorgt dat de aanvragen voor reproductie worden
afgehandeld}}
\newglossaryentry{def:uitleenstatus}{name=Uitleenstatus,text=uitleenstatus,description={Een
status van een stuk dat aangeeft waar een bepaald exemplaar van een stuk zich op dit moment bevindt.
Mogelijke waarden zijn: Beschikbaar, Aangevraagd, Uitgeleend, Teruggebracht}}
\newglossaryentry{def:wachtnummer}{name=Wachtnummertje,text=wachtnummertje,description={Een
nummer dat kan worden gebruikt om een bezoeker in een wachtrij te plaatsen. Als
de bezoeker een aanvraag ter inzage doet krijgt hij/zij dit nummer. Als de
stukken uit het magazijn zijn gehaald wordt het wachtnummer meegedeeld en kan de
bezoeker de stukken komen ophalen}}
\newglossaryentry{def:iDeal}{name=iDeal,description={Veelgebruikte Nederlandse
online betalingsmethode}}
\newglossaryentry{def:contactpersoon}{name=Contactpersoon,text=contactpersoon,description={Persoon
waarmee contact dient te worden opgenomen indien er een inzage restrictie op een
archief(stuk) rust}}
\newglossaryentry{def:metadata}{name=Metadata,text=metadata,description={Gegevens
over andere data. Denk aan de uitleenstatus, of inzage restrictie die een stuk
kan hebben}}
\newglossaryentry{def:inzage-restrictie}{name=Inzage Restrictie,text=inzage
restrictie,description={Restrictie wat betreft de inzage. Stukken kunnen `Open'
zijn, dan mag iedereen ze inzien, en dan mogen er ook reproducties van worden
gemaakt. `Restricted' geeft aan dat een stuk beperkt mag worden ingezien, de
exacte beperkingen zijn in de algemene metadata van een stuk vastgelegd (niet
per exemplaar). `Closed' geeft aan dat de stukken helemaal niet in mogen worden
gezien}}
\newglossaryentry{def:gebruiksrestrictie}{name=Gebruiksrestrictie,text=gebruiksrestrictie,
description={Beperking wat betreft het gebruik van een stuk. Het kan
bijvoorbeeld zijn dat alleen de microfilm mag worden uitgeleend en het origineel
niet. Dit is aangegeven per exemplaar}}
\newglossaryentry{def:embargodatum}{name=Embargodatum,text=embargodatum,description={Datum
waarna een stuk vrij wordt gegeven voor inzage. De inzage restrictie zal dan
automatisch naar `Open' veranderen}}
\newglossaryentry{def:framework}{name=Framework,text=framework,description={Een
raamwerk aan software componenten dat standaard-oplossingen biedt voor bepaalde
handelingen die vaak worden uitgevoerd bij het schrijven van software.
Voorbeeld: Django is een framework dat standaard-oplossingen biedt voor het
schrijven van een web-applicatie in de programmeertaal Python}}
\newglossaryentry{def:library}{name=Library,text=library,plural=libraries,
description={Een bibliotheek aan software componenten, welke gebruikt kan worden
bij het schrijven van andere software}}
\newglossaryentry{def:open-source}{name=Open Source,text=open
source,description={Een term gebruikt om aan te duiden dat de broncode van de
software voor iedereen toegankelijk is}}
\newglossaryentry{def:web-interface}{name=Web Interface,text=web
interface,description={User interface toegankelijk via een webbrowser, \emph{zie
User Interface}}}
\newglossaryentry{def:aanvraag}{name=Aanvraag,text=aanvraag,plural=aanvragen,description={Een
bezoeker kan een aanvraag doen om bepaalde stukken in te zien. Een aanvraag is
dus een verzameling stukken die gereserveerd danwel uitgeleend zijn voor inzage
door een bezoeker}}
\newglossaryentry{def:rechthebbende}{name=Rechthebbende,text=rechthebbende,description={De
persoon die eigenaar is van een verzameling van stukken/archief. Dit is in vele
gevallen ook de contactpersoon}}
\newglossaryentry{def:actor}{name=Actor,text=actor,description={Een type
gebruiker van het systeem (bijv. bezoeker of studiezaalmedewerker)}}
\newglossaryentry{def:reservering}{name=Reservering,text=reservering,plural=reserveringen,
description={\emph{Zie Aanvraag}}}
\newglossaryentry{def:preconditie}{name=Preconditie,text=preconditie,
description={Gegeven conditie die geldt voor aanvang van een reeks acties
(zoals een use case)}}
\newglossaryentry{def:postconditie}{name=Postconditie,text=postconditie,
description={Gegeven conditie die geldt na het uitvoeren van een reeks
acties (zoals een use case)}}
\newglossaryentry{def:api}{name=API,description={Application Programming
Interface; Een verzameling functies om de services die een bibliotheek of ander
programma biedt te kunnen gebruiken/aanroepen}}
\newglossaryentry{def:mvc}{name=MVC,description={Model-View-Controller;
Ontwerppatroon waarbij een duidelijke scheiding is gemaakt tussen de stukken
code die de data beheert (Model), de stukken code die zorgen voor de
presentatie(View), en de code die de logica van het programma bevat (Controller)}}
\newglossaryentry{def:vufind}{name=VuFind,description={Zoeksysteem veelal
gebruikt om catalogi te doorzoeken}}
\newglossaryentry{def:oai}{name=OAI-PMH,text=OAI,description={Open Archives
Initiative Protocol for Metadata Harvesting; Protocol om metadata tussen grote
archiefinstellingen uit te wisselen. Dit protocol is ge\"implementeerd door het
IISG (api.iisg.nl)}}
\newglossaryentry{def:sor}{name=SOR,description={Shared Object Repository; Een
extern systeem waar gedigitaliseerde stukken in worden opgeslagen}}
\newglossaryentry{def:rest}{name=REST,description={Representational State
Transfer; Een architecturele stijl om websites op te bouwen. Per URL kunnen
maximaal 4 verschillende request methoden worden gebruikt: GET om data van een
bepaalde URL locatie op te vragen, POST om data onder de gespecificeerde URL aan
te maken, PUT om data op de exact opgegeven locatie aan te maken/overschrijven,
en DELETE om de data op de locatie te verwijderen}}
\newglossaryentry{def:json}{name=JSON, description={JavaScript Object Notation;
Manier van noteren van Javascript objecten}}
\newglossaryentry{def:pid}{name=PID,description={Persistant Identifier; Een
persistente unieke code die gebruikt kan worden om bepaalde objecten (in dit
geval: stukken) te identificeren. Doordat het in zekere zin globale identifiers
zijn, kunnen verschillende systemen deze gebruiken om naar hetzelfde object te
refereren}}
\newglossaryentry{def:jsonp}{name=JSONP,description={JSON with Padding; Een
manier om data uit te wisselen tussen verschillende websites. Site A doet een
aanvraag naar Site B, met een callback functie als parameter. Site B geeft een
JSON object terug, met de callback als omhulsel. Site A voert de callback
functie met het JSON object als parameter uit}}
\newglossaryentry{def:info-vel}{name=Informatievel,text=informatievel,
plural=informatievellen,
description={Vel papier met informatie (locatie in magazijn, titel, datum
aanvraag e.d.) van een aangevraagd stuk erop.
Dit vel wordt aan de bezoeker overhandigd zodra deze een stuk in ziet. Als de
bezoeker een stuk terugbrengt kan de magazijnmedewerker aan de hand van dit vel
zien op welke plekken in het archief de stukken terug moeten worden gezet}}
\newglossaryentry{def:plaats-vel}{name=Plaatsvervangingsvel,
text=plaatsvervangingsvel, description={Vel papier met informatie (locatie in
magazijn, titel, datum aanvraag e.d.) van een aangevraagd stuk erop. Dit vel
wordt op de plek waar een stuk in het magazijn stond gelegd als plaatsvervanger.
Als het stuk terug wordt gebracht is zo gemakkelijk te zien waar het stuk op
plank-niveau precies hoort te staan}}
\newglossaryentry{def:milestone}{name=Milestone,text=Milestone,
description={Een periode waarin een groot aantal nieuwe dingen wordt toegevoegd
aan een software programma. Een milestone kan uit meerdere iteraties bestaan om
de tijd tussen implementatie en feedback te verkorten. Na elke milestone wordt
een nieuwe versie van de software opgeleverd, met daarin veel nieuwe
toevoegingen. Een milestone duurt meestal een aantal maanden tot een jaar,
afhankelijk van de omvang van het project}}
\newglossaryentry{def:iteratie}{name=Iteratie,text=iteratie, description={Een periode
waarin een beperkt aantal nieuwe dingen wordt toegevoegd aan een programma, of
waarin fouten worden opgelost. Een iteratie duurt meestal een aantal weken tot
een maand, afhankelijk van het project, en is onderdeel van een milestone. Na
elke iteratie wordt vaak een evaluatie gehouden om het project indien nodig een
andere koers te geven}}
\newglossaryentry{def:orm}{name=ORM, description={Object Relational Mapper; Een
stuk software dat ervoor zorgt dat objecten uit een Object Geori\"enteerd
programma aan tabellen uit een relationele database worden gekoppeld}}
\newglossaryentry{def:moscow}{name=MoSCoW, description={Must have, Should have,
Could have, Won't have but would like to have; Een prioritiseringsmethode
gebruikt om aan te geven welke onderdelen essentieel zijn en welke onderdelen
minder belangrijk zijn voor het slagen van een implementatiefase in een
(software) project. De must-haves zijn essentieel, de should-haves zijn niet
essentieel maar het zou jammer zijn als deze onderdelen niet werden
ge\"implementeerd. De could-haves zijn optioneel, en worden alleen
ge\"implementeerd als er nog tijd over is. De won't-haves worden niet
ge\"implementeerd omdat ze niet haalbaar zijn binnen het gestelde tijdsbestek}}
